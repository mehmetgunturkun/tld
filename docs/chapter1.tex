\documentclass{article}

\begin{document}
    \section{TLD Implementattion}
        \paragraph{}
            TLD implementation is based on three main modules such as: Tracker, Detector and Learner (Integrator).
            Tracker and Detector modules are completely independent from each other.
            However, Integrator combines two separate results from Tracker and Detector, validates and
            sends feedback to detector in order to rule out false positives and false negatives.
        \subsection{Tracker}
            \paragraph{}
                Tracker takes two consequent frames and a bounding box contains the target object within the first image.
                Trackers main task is that to estimate the location of the bounding box in the current frame or to decide whether the
                target object is still visible. Basically, tracker generates points in the given bounding box,
                tracks each point from frame to frame, filter the reliable ones and estimates the overall displacement
                with reliable points.
            \paragraph{Point Generation}
                In order to generate points in the given box, grid point approach is used. With a margin to each border of box,
                put a grid with 10 grid points in both direction: width and height.
            \paragraph{Point Tracking}
                TLD uses pyramidal lucas-kanade tracker, which is already implemented in OpenCV, to track each point from frame
                to frame; but instead of tracking one way, we track the points, previous operation has end up, backwards.
                This forward-backward tracking gives us a good metric for evaluation of reliability of points.
                \textbf{I can explain flags and parameters in calcOpticalFlowPyrLK?}
            \paragraph{Filtering Reliable Points}
                At this point, we have point to point correspondences with forward backward error. However, we still do not know which points
                are really tracked successfully. This is why, we apply normalized cross correlation between points to decide that
                still coordinates point to same location of object or shifted away. Besides ncc error, we already have forward-backward error.
                With these error metrics, we apply median filtering to points and eliminate outliers.
            \paragraph{Relocation of Box}
                At the end of the tracking process, we have points and relatively robust displacements for each of them.
                With these displacements, we, again, get the median value of displacements in two direction and
                to the coordinates of box respectively.
        \subsection{Detector}
            \paragraph{}
                Detector also takes the current frame and returns couple of bounding boxes
                which are probably includes the instances of the object. Since cascaded model is used in
                design of the detector, there are multi layers that detector consist of.
                \begin{itemize}
                    \item What is the overall flow to generate possible boxes from given frame?
                    \item What are the layers/steps in cascaded detector?
                    \item What are responsibilities and aim of the each step?
                \end{itemize}
        \subsection{Learner}
            \paragraph{}
                Learners job is to reason out the tracking result and detection result.
                \begin{itemize}
                    \item How to validate when Tracker tracks given box?
                    \item How to decide whether detector or tracker is valid?
                    \item All possible cases, etc?
                \end{itemize}
\end{document}
